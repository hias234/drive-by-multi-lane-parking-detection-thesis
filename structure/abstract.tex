%\chapter*{Summary}
%\markright{Summary}

%Summary \ldots

\chapter*{Abstract}
\markright{Abstract}

Traffic congestion is one of the largest problems in urban areas nowadays. Not only the increased number of daily trips contributes to traffic jams, but also drivers who are looking for parking spaces. When driving around a long time to find vacant parking spaces, vehicles not only congest the streets, but also emit a significant amount of green house gases. To mitigate these problems, increased transparency of the current parking space availability situation is desirable to help drivers navigate directly to vacant parking spaces close to their end destination. 

This thesis tackles the task of creating parking space availability maps for cities to make it easier for drivers to find vacant parking spaces. A drive-by approach is chosen to detect vacant and occupied parking spaces. 
Using this approach, several vehicles are driving through the city and are continuously collecting sensor data. All vehicles measure the distance to the right side of the road as well as their position using GPS. Using the distance information, our prototype should be able to decide whether a parking space is occupied or vacant. Furthermore, distracting situations which prevent a successful detection (e.g. an overtaking situation) should be detected as well. 
A prototype vehicle has been equipped with the necessary sensors. Real world sensor data has been collected during 32 test drives and is used in the evaluation process to identify the best performing systems.

As the traffic situations in our test drives can be fairly complicated, it is non-trivial to find a rule set which can detect the different situations. Thus, machine learning techniques are tested on their performance. Using a manually tagged dataset, containing nine computed features derived from the sensor data, several traditional machine learning algorithms and deep learning approaches are evaluated on their classification performance. Furthermore, several approaches to further improve the classification performance are tested, such as over- and under-sampling. 
Finally, a custom two-staged classification process is introduced to include information of the surroundings of the parking situation to classify. This approaches reaches the best classification results. In total, 96.52\% of the situations are correctly classified and 93.81\% of the parking cars are detected.


