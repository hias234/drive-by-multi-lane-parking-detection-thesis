%\chapter*{Summary}
%\markright{Summary}

%Summary \ldots

\chapter*{Abstract}
\markright{Abstract}

Traffic congestion is one of the largest problems in urban areas nowadays. Not only the increased number of daily trips contributes to traffic jams, but also drivers who are looking for parking spaces. When driving around a long time to find vacant parking spaces, vehicles not only congest the streets, but also emit a significant amount of green house gases. To mitigate these problems, increased transparency of the current parking space availability situation is desirable to help drivers navigate directly to vacant parking spaces close to their end destination. 

This thesis tackles the task of creating parking space availability maps for cities to decrease parking space search times. A drive-by approach is chosen to detect vacant and occupied parking spaces. This approach leverages several vehicles that are driving through the city and are continuously collecting sensor data. All vehicles measure the distance to the right side of the road as well as their position. Using the distance information, our system decides whether a parking space is occupied or vacant. Furthermore, distracting situations which prevent a successful detection, such as an overtaking situation, need to be detected as well. 
To demonstrate the feasibility of drive-by parking space sensing, a prototype is developed that employs a LIDAR sensor for optical distance measurements, GPS for positioning, and a parking space detection software running on a Raspberry Pi computer.
For testing, a vehicle is equipped with the prototype system. Real world sensor data have been collected during 32 test drives. In total 444.427 distance measurements as well as 14.997 GPS measurements have been sensed.

As road traffic situations show different and complicated patterns, it is non-trivial to find a rule set which can detect the different parking situations. Thus, supervised machine learning techniques are employed and tested with respect to their classification performance. Using a manually tagged dataset containing nine computed features derived from the sensor data, several traditional machine learning algorithms and deep learning approaches are evaluated and compared with one another. Furthermore, filtering irrelevant samples as well as over- and under-sampling are tested to further improve the classification performance. 
Finally, a custom two-staged classification process is introduced to include information of the surroundings of the parking situation in addition to its own features. This approaches reaches the best classification results. In total, 96.52\% of the situations are correctly classified and 93.81\% of the parking cars are detected.


