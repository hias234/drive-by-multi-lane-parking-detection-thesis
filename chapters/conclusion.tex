\chapter{Conclusions and Future Work}
\label{chap:conclusion}



This thesis focuses on sensing a city's parking space availability to increase the transparency of the current city-wide parking situation. 
Known parking space availability can help drivers, searching for vacant parking spaces, to directly navigate to a spot and avoid unnecessary driving. This can reduce traffic congestion, green house gas emissions, and driver frustration. 
In this work, a prototype using a drive-by parking space detection approach has been implemented with the goal to create accurate parking space availability maps. The prototype demonstrates the feasibility of a drive-by parking space detection system and allows to study the achievable detection accuracy. Especially the performance of such a system in real world traffic scenarios on streets with multiple lanes and surrounding traffic is of interest.

To analyze the current state of the art, several existing parking space detection approaches have been investigated and compared to each other to find advantages and disadvantages.
Many of the existing approaches deliver promising results. However, most of them would either cause dramatically high costs due to sensor installations or would need a large user base reporting free spaces, e.g. via a smartphone app.
Drive-by parking space sensing approaches, which use distance sensors, have been found as promising. However, all of the reference works are evaluated only on simple traffic scenarios without any distractions, such as overtaking situations. 
The goal of this thesis is to investigate if a similar accuracy can be achieved on real world traffic data.

As a first step, data from real world traffic has been gathered. A prototype system, which collects the necessary sensor data, has been designed, built, and mounted on a test vehicle. The used sensors (a LIDAR optical distance sensor and a common GPS module) have been selected to be able to analyze complex traffic scenarios while still coming at a low price. 
During 32 test drives in Linz, sensor data has been collected making up a dataset containing all kinds of different situations in real world traffic. 
In total 444.427 distance measurements as well as 14.997 GPS measurements have been sensed.
To be able to automatically evaluate the dataset, it has been pre-processed, segmented, and the ground truth has been labelled manually. The traffic situations used for labelling are \emph{free space}, \emph{parking car}, \emph{overtaking situation}, and \emph{other parking vehicle}.
%Testing our acquired dataset with an existing approach of drive-by sensing \cite{Mathur:2010:PDS:1814433.1814448}, the performance turned out to be much worse than with the dataset used in their work (see Section \ref{sec:features_other_pds}). The author concludes that due to our more complex dataset, which contains multi lane scenes as well as overtaking situations, the performance of the reference implementation would not be sufficient to lead to satisfying outcome on our dataset.

To determine the accuracy which can be achieved using drive-by parking space sensing in complicated traffic situations, machine learning- as well as deep learning algorithms have been evaluated. While deep learning algorithms are operating on raw sensor data, for traditional machine learning methods, nine features have been introduced. The experiments show that traditional machine learning approaches outperform deep learning on parking space sensing. A random forest classifier gains the best results with an overall accuracy of about 96\% and a recall on parking cars of about 88\%. Experiments with all techniques also indicate that distractions, such as overtaking situations, are only detected at a very low rate. This is likely caused by the low amount of such samples in the acquired dataset, but also by their similarity to other parking situations.

In addition to the basic methods, several techniques to improve the classification performance have been tested. First, a map of parking space areas in Linz has been created by clustering the parking spaces, included in the dataset. This map has been used to identify relevant samples in the dataset that are close to parking space areas. A random forest classifier using a dataset containing only relevant samples detects about 4\% more parking cars than using the full dataset. Another technique which improved the results was to include the surroundings of a parking situation when classifying it. Using a two-staged classification process with two random forests, the accuracy could be further improved, yielding an overall accuracy of about 96.5\% and a recall of parking cars of about 93.8\%.

The results indicate that, using the proposed sensor setup in conjunction with machine learning, drive-by sensing can operate reliably also in complex scenes in real world traffic. 
A study of Mathur et al. about drive-by parking space sensing \cite{Mathur:2010:PDS:1814433.1814448} shows that an achieved accuracy of 87.6\% on detecting parking cars is sufficient to create accurate parking space availability maps. With our best result of 96.5\%, we could achieve an even better accuracy.
Yet, distractions, such as overtaking situations, are detected only at a low rate of about 30\%. However, these situations appear quite rarely.

In Future smart cities, parking space availability maps are vital knowledge to improve city-traffic and to avoid unnecessary cruising.
While this work shows that it is possible to detect parking cars while driving by, future research has to tackle the following problems. Although our dataset allows to derive valuable insights, a larger dataset is desired to increase the detection rate for overtaking situations and also might further improve the classification results. Another future research question is how many sensing vehicles would be needed in cities of different sizes and topologies to gain high quality parking space maps. Simulations may help to gain insight into this topic. Self-driving vehicles will lead to further research topics. Fully autonomous vehicles might be able to search vacant parking spaces and park on their own, coordinating their activities which allows to introduce sophisticated parking space reservation systems.









