\chapter{Conclusions and Future Work}
\label{chap:conclusion}



This thesis focuses on sensing a city's parking space availability to increase the transparency of the current city-wide parking situation. 
A known parking space availability can help drivers, which are searching for vacant parking spaces, to directly navigate to a vacant spot and avoid driving around blocks. This can reduce traffic congestion, green house gas emissions, and driver frustration. 
In this work, a prototype using a drive-by parking space detection approach has been implemented with the goal to create accurate parking space availability maps. The results of the prototype should show at what accuracy drive-by detection systems can operate. Especially the performance of such a system in real world traffic scenarios on streets with multiple lanes and a lot of traffic is of interest.
This chapter summarizes the most essential aspects of this work, its results, and its most important conclusions.



To analyze the current state of the art, several existing parking space detection approaches have been investigated and compared to each other to find advantages and disadvantages.
Many of them deliver promising results. However, most of them would either cause dramatically high costs (e.g. by requiring, installing and maintaining thousands of sensors) or would need a high user base (e.g. a lot of people would need a smartphone app). 
 Drive-by park sensing approaches, which use 1D distance sensors, have been found as promising. Using these approaches several sensing vehicles drive through the city and constantly sense the distance to the nearest obstacle on the right side of the road (in many cases parking cars). This information in conjunction with GPS data is then used to detect parking cars. However, all of the reference works are evaluated only on simple traffic scenarios and without any distractions, such as overtaking situations. 
The goal of this thesis is to investigate if a similar accuracy can be achieved on real world traffic data.

As a first step, data from real world traffic has been gathered. A prototype system, which collects the necessary sensor data, has been designed, built, and mounted on a test vehicle. The used sensors have been selected to be able to analyze complex traffic scenarios while still coming at a low price. 
%The total costs of the system are about \euro{240}. 
In 32 test drives in Linz, sensor data has been collected making up a dataset containing all kinds of different situations in real world traffic. To be able to automatically evaluate the dataset, it has been pre-processed, segmented and the ground truth has been tagged manually. 
Testing our acquired dataset with an existing approach of drive-by sensing \cite{Mathur:2010:PDS:1814433.1814448}, the performance turned out to be much worse than with the dataset used in their work (see Section \ref{sec:features_other_pds}). The author concludes that due to our more complex dataset, which contains multi lane scenes as well as overtaking situations, the performance of the reference implementation would not be sufficient to lead to satisfying outcome on our dataset.

%The author concludes that this can be explained due to the more complex parking situations in the dataset, which was acquired in this work. 
%Therefore, the goal of acquiring a complex dataset is achieved.

%Tests on the processed dataset led to the conclusion that the data contained much more variety than similar datasets used in other related work, making it harder to classify the situations correctly (see Section \ref{sec:features_other_pds}). 

To determine the accuracy which can be achieved using drive-by park sensing on complex data, machine learning- as well as deep learning algorithms have been evaluated on their performance. While deep learning algorithms are operating on raw sensor data, for traditional machine learning methods, nine computed features have been introduced. The experiments show that traditional machine learning approaches outperform deep learning on park sensing. A random forest classifier gains the best results with an overall accuracy of about 96\% and a recall on parking cars of about 88\%. Experiments with all techniques also indicate that distractions, such as overtaking situations, are only detected at a very low rate. This is probably caused due to the low amount of such samples in the acquired dataset. 
%Sampling techniques can improve the performance of detecting overtaking situations. However, they also worsen the detection rate of parking cars which is much more important. Thus, sampling techniques are not considered to help achieving the overall goal.

Additionally to the general experiments, several techniques to improve the classification performance have been tested. First, a map of parking space areas in Linz has been created by clustering the parking spaces, included in the dataset. This map has been used to select relevant samples of the dataset (the samples which are close to parking space areas). A random forest classifier using a dataset containing only relevant samples detects about 4\% more parking cars than using the full dataset. Another technique which improved the results was to include the surroundings of a parking situation when classifying it. Using a two-staged classification process with two random forests, the accuracy could be further improved, having an overall accuracy of about 96.5\% and a recall of parking cars of about 93.8\%.

The results indicate that, using the proposed sensor setup in conjunction with machine learning, drive-by sensing can operate at a high performance also in complex scenes in real world traffic. 
A study of Mathur et al. about drive-by park sensing \cite{Mathur:2010:PDS:1814433.1814448} suggests that the achieved accuracy of 93.8\% on detecting parking cars is sufficient to create accurate parking space availability maps.
Distractions, such as overtaking situations, are detected only at a low rate (about 30\%). However, they only appear quite rarely. Thus, the low detection rate is of minor importance, when deriving a city's parking space situation and does not alter the end result noticeably.

%Future work
\section{Future Work}
Due to an increasing number of cars driving in cities, detecting the parking space availability will be an increasingly important task in the future to help drivers find vacant parking spaces easily and to reduce traffic induced by parking space search. While this work shows that it is possible to detect parking cars while driving by, future research has to tackle several following problems. A following research topic to this work would be to acquire a bigger dataset which might help to increase the detection rate for overtaking situations and also might further improve the classification results. Another future research question is how many sensing vehicles would be needed in cities of different sizes and topologies to gain high quality parking space maps. Simulations could be created to gain insight into this topic. Self-driving vehicles provide further research topics, possibly leading to fully autonomous vehicles, which might be able to search vacant parking spaces and park on their own.









