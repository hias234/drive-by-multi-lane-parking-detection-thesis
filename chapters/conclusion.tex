\chapter{Conclusions and Future Work}
\label{chap:conclusion}



This thesis focuses on sensing a city's parking space availability to increase the transparency of the current city-wide parking situation. 
A known parking space availability can help drivers, which are searching for vacant parking spaces, to directly navigate to a vacant spot and avoid driving around blocks. This can reduce traffic congestion, green house gas emissions, and driver frustration. 
In this work, a prototype using a drive-by parking space detection approach has been implemented with the goal to create accurate parking space availability maps. The results of the prototype should show at what accuracy drive-by detection systems can operate. Especially the performance of such a system in real world traffic scenarios on streets with multiple lanes and a lot of traffic is of interest.
The following paragraphs summarize the most essential aspects of this work, its results, and its most important conclusions.



To analyze the current state of the art, several existing parking space detection approaches have been investigated and compared to each other to find advantages and disadvantages.
Many of them deliver promising results. However, most of them would either cause dramatically high costs (e.g. by requiring, installing and maintaining thousands of sensors) or would need a high user base (e.g. a lot of people would need a smartphone app). 
 Drive-by park sensing approaches, which use 1D distance sensors, have been found as promising. However, all of the reference works are evaluated only on simple scenarios without any distractions, such as overtaking situations. 
The goal in this thesis is to investigate if a similar accuracy can be achieved on real world traffic data.

As a first step, data from real world traffic has been gathered. A prototype system, which collects the necessary sensor data, has been designed, built, and mounted on a test vehicle. The used sensors have been selected to be able to analyze complex traffic scenarios while still coming at a low price. 
%The total costs of the system are about \euro{240}. 
In 32 test drives in Linz, sensor data (distance- and GPS-data) has been collected making up a dataset containing all kinds of different situations in real world traffic. To be able to automatically evaluate the dataset, it has been pre-processed, segmented and the ground truth has been tagged manually. Tests on the processed dataset led to the conclusion that the data contained much more variety than similar datasets used in other related work investigating drive-by approaches (see Section \ref{sec:features_other_pds}). 









