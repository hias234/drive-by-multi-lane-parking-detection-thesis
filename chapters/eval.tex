\chapter{Results and Discussion}
\label{chap:evaluation}

This chapter summarizes the results of the machine learning experiments and compares different machine learning models and datasets on their performance. First of all, Section \ref{sec:optimization_goal} discusses the measures which should be optimized to derive most accurate parking space maps. Furthermore, Section \ref{sec:machine_learning_results} shows the results of classical machine learning model, while Section \ref{sec:deep_learning_results} will discuss the results of the deep learning neural networks operating on raw sensor data.






\section{Optimization Goal}
\label{sec:optimization_goal}

This section should outline the processes and decisions which have been taken to determine the best performing system which would produce the most accurate parking space maps. Of course, the first interesting measure when looking at machine learning experiments is the accuracy. However, it is not always the most important one, when it comes to solving a specific problem. 

In the parking space detection scenario, which is tackled in this thesis, the overall goal is to derive accurate parking space maps using drive-by park sensing. The first step to achieve this goal is to detect parking cars, because only then it can be decided if a spot is vacant or occupied at a specific time (detecting a \emph{free space}-class is not enough as it does not determine if it is allowed to park at this spot). Therefore, the most important measure, which should be optimized, are precision and recall of the \emph{parking car}-class. As secondary goals, precision- and recall-measures of the classes \emph{overtaking situation} and \emph{other parking vehicle} should be optimized. However, these classes have only a minor occurance in the dataset and therefore, the performance is expected to be much worse than for the \emph{parking car}-class.

To be able to compare two classification results containing recall- and precision measurements, the $f-measure$ (or also called $F_1-score$) has been chosen as comparison measurement. Because it is the harmonic mean of both recall and precision it is well suited for the task. The formula to calculate the $f-measure$ is: $F_1 = 2 \times \frac{precision \times recall}{precision + recall}$.






\section{Machine Learning Results}
\label{sec:machine_learning_results}

This section discusses the results of the machine learning experiments with common machine learning models using computed feature values as input. As evaluation-method 10-fold cross validation has been chosen. Figure ... shows the comparison of several machine learning models with different parameters in terms of accuracy, precision and recall of the "parking car"-class and the necessary runtime to learn the model.


\begin{table}


\resizebox{\textwidth}{!}{%
\centering
\bgroup
\def\arraystretch{1.4}
\begin{tabular}{| r || c | c | c |}
\hline
	&
   \textbf{Accuracy} & 
   \textbf{Recall Parking Car} &
   \textbf{Precision Parking Car} \\
   &
	(full / filtered dataset) & 
	(full / filtered dataset) &
	(full / filtered dataset) \\
\hline
  \textbf{Random Forest} & 
   0.361 / 0.502 &
   0.0 / 0.0 &
   0.0 / 0.0 \\
\hline
  \textbf{Decision Tree} & 
   0.361 / 0.502 &
   0.0 / 0.0 &
   0.0 / 0.0 \\
\hline
  \textbf{Neural Network} & 
   0.361 / 0.502 &
   0.0 / 0.0 &
   0.0 / 0.0 \\
\hline
  \textbf{Naive Bayes} & 
   0.361 / 0.502 &
   0.0 / 0.0 &
   0.0 / 0.0 \\
\hline
  \textbf{kNN classifier} & 
   0.361 / 0.502 &
   0.0 / 0.0 &
   0.0 / 0.0 \\
\hline
  \textbf{Support Vector Machine} & 
   0.361 / 0.502 &
   0.0 / 0.0 &
   0.0 / 0.0 \\
\hline

\end{tabular}
\egroup
}

\caption{Results of classic machine learning models applied on the full and filtered dataset.}
\label{table:classic_ml_results}
\end{table}


\begin{table}


\resizebox{\textwidth}{!}{%
\centering
\bgroup
\def\arraystretch{1.4}
\begin{tabular}{| r || c | c | c | c |}
\hline
   Predicted class $\rightarrow$ &
   \textbf{Free Space} & 
   \textbf{Parking Car} &
   \textbf{Overtaking} &
   \textbf{Other Parking} \\
   True class $\downarrow$ &
	 & 
	 &
	 \textbf{Situation} &
	 \textbf{Vehicle} \\
\hline
  \textbf{Free Space} & 
   0.361 / 0.502 &
   0.0 / 0.0 &
   0.0 / 0.0 & \\
\hline
  \textbf{Parking Car} & 
   0.361 / 0.502 &
   0.0 / 0.0 &
   0.0 / 0.0 & \\
\hline
  \textbf{Overtaking Situation} & 
   0.361 / 0.502 &
   0.0 / 0.0 &
   0.0 / 0.0 & \\
\hline
  \textbf{Other Parking Vehicle} & 
   0.361 / 0.502 &
   0.0 / 0.0 &
   0.0 / 0.0 & \\
\hline

\end{tabular}
\egroup
}

\caption{Confusion Matrix.}
\label{table:best_clf_confusion_matrix}
\end{table}


Comparison of all tested algorithms -> accuracy -> different feature sets -> Graphics

Description of Graphic

Maybe binary classification

Filtered Dataset -> Parking Space Maps

Stacked Classifier


\section{Deep Learning Results}
\label{sec:deep_learning_results}

Same as Machine Learning REsults






