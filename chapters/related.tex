\chapter{Related Work}
\label{chap:relatedwork}

This chapter will cover work related to park sensing and machine learning. Section \ref{sec:parksensing} will discuss different approaches to sensing current parking situations in cities. Furthermore, a comparison of them will be given as well as advantages and disadvantages of the specific approaches. \todo{Machine Learning...}



\section{Approaches to Park Sensing}
\label{sec:parksensing}

There already exist numerous approaches to detecting the states of parking spaces as well as the parking situation in a city. Parking detection approaches can coarsely categorized in four different categories. In the following subsections several reference papers for all approaches will be discussed. 

\subsection{Stationary Park Sensing}
\label{sec:stationary_park_sensing}

SFPark, London, Streetline, Google Open Spot

multi-classifier image detection system\cite{stationary_camera_sensing}

arm smart camera \cite{stationary_camera_sensing_arm_smart_camera}

\subsection{Counting Vehicles}
\label{sec:counting_in_out_park_sensing}
smart urban parking detection \cite{smarturbanparkingdetection}

\subsection{Event Detection based Park Sensing using Smartphones}
\label{sec:event_detection_park_sensing}

updetection\cite{Ma:2014:USP:2674918.2674929}

parsense \cite{Nawaz:2013:PSB:2500423.2500438}

pocketparker \cite{Nandugudi:2014:PPP:2632048.2632098}

\subsection{Drive-by Park Sensing using Cameras}
\label{sec:related_driveby_park_sensing_distance}

\subsubsection{ParkMaster}
 
parkmaster \cite{Grassi:2017:PIE:3132211.3134452}


\subsection{Drive-by Park Sensing using Distance Sensors}
\label{sec:related_driveby_park_sensing_distance}

\subsubsection{ParkNet}
\label{sec:parknet}

parknet... \cite{Mathur:2010:PDS:1814433.1814448}







\section{Machine Learning}

