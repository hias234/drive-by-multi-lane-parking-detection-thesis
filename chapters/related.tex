\chapter{Related Work}
\label{chap:relatedwork}

This chapter will cover work related to park sensing and machine learning. Section \ref{sec:parksensing} will discuss different approaches to sensing current parking situations in cities. Furthermore, a comparison of them will be given as well as advantages and disadvantages of the specific approaches. \todo{Machine Learning...}



\section{Approaches to Park Sensing}
\label{sec:parksensing}

There already exist numerous approaches to detecting the states of parking spaces as well as the parking situation in a city. In this section parking detection approaches will be categorized in five different categories and in the following subsections several reference papers for all categories will be discussed. The first section about stationary park sensing (section \ref{sec:stationary_park_sensing}) will reference approaches where sensors are stationary deployed per parking space. Section \ref{sec:stationary_park_sensing_cameras} will cover camera-based systems which detect parking spaces in a dedicated area. Counting in- and outgoing vehicles to coarsely estimate parking space counts is discussed in section \ref{sec:counting_in_out_park_sensing}. Another approach is to detect certain events (for instance parking and unparking) to estimate parking space counts. Section \ref{sec:event_detection_park_sensing} will discuss several related papers in this area. Drive-by sensing will be discussed in sections \ref{sec:related_driveby_park_sensing_cameras} and \ref{sec:related_driveby_park_sensing_distance}. Using cameras to detect parking spaces while driving by will be referenced in section \ref{sec:related_driveby_park_sensing_cameras}, while section \ref{sec:related_driveby_park_sensing_distance} will cover using distance sensors. \todo{evtl. comparison, advantages, disadvantages section}



\subsection{Stationary Park Sensing}
\label{sec:stationary_park_sensing}

The most obvious and technically simple solution to park sensing is to use stationary sensors to determine the state of parking spaces. Usually one sensor per parking space is used which determines its state (occupied or vacant) and sends it to a central server. Several reference projects already exist \cite{SFPark, VehicleSense} implementing this technique. In this section as an example of all similar systems, the SFpark project will be examined.

In San Francisco a first prototype of the SFpark project \cite{SFPark} has been implemented from 2008 to 2011. In specific down town areas in San Francisco with high amounts of traffic, wireless stationary sensors have been placed at about 8.200 road side parking spaces. These sensors are able to detect the state of one parking space in real time and send this information to a central server. Furthermore, parking garages also count the in- and outgoing vehicles and shared this information, so that the parking situation in the areas with park sensing can be derived. The gained information is being shared as open data, so third party developers and researchers can also use the dataset for all kind of projects. Furthermore, there exists an App from SFpark itself to help drivers find available nearby parking spaces, navigate to it and pay as they go with their phones. 

A top priority goal of the SFpark project is to increase the availability of parking spaces in every block throughout the city. To achieve this goal, they are using demand responsive pricing. If (almost) all parking spaces in a neighbourhood are occupied for a long time, they raise the price in this specific area and vice versa if no parking spaces are occupied they lower prices. This leads to high overall parking space availabilities (20 - 40\%) and also to lower traffic congestion and lower greenhouse gas emissions. However, besides the advantages of high accuracy and being a real time system, there also are disadvantages. First of all, only metered parking spaces can be tracked, as sensors have to be installed per parking space, so areas where parking is allowed but there are no clearly marked parking spaces cannot be sensed with a reasonable accuracy. Furthermore, another big drawback are the high overall costs. The about 8.200 stationary sensors have to be bought, installed and maintained, which obviously causes high costs while only covering a tiny fraction of the overall San Francisco down town. So if the system should be available in the whole down town, costs would increase dramatically.

%Google Open Spot, London,





\subsection{Stationary Park Sensing using Cameras}
\label{sec:stationary_park_sensing_cameras}

Another approach while using stationary sensors is to use fixed deployed cameras which continuously record images of parking areas and analyze it for vacant parking spaces. Cameras can monitor up to one hundred parking spaces simultaneously with an accuracy up to 96 \%. Challenges of image detection are of course different lightning and weather conditions as well as occlusions depending on the angle which the camera records the parking scene. Detection algorithms using standard digital image processing will be discussed in this section as well as approaches using deep learning and Convolutional Neural Networks (CNNs).

\paragraph{Parking Detection using Digital Image Processing}

There exist a few common approaches using digital image processing to detect the state of parking spaces using a captured image of a parking area. First of all, edge detection is often used for parking space classification. Common edge detectors such as Canny Edge Detector or Sobel can be used to derive the edge pixels of an image. As next step the edges or edge pixels are counted and if they are above a certain threshold, the space will be detected as occupied. The assumption behind this approach is, that usually a vacant parking space has a plain surface and therefore a low amount of edges whereas a image of a parking car should have a lot of edges. Blumer et al. \cite{Blumer2012} and Liu et al. \cite{stationary_camera_sensing} both used this technique as part of their algorithms. 

Often a slightly different yet related approach is also used, namely object counting \cite{stationary_camera_sensing}. The edges of the image segment of a parking space are analyzed and closed contours (treated as objects) are detected and counted. Then again depending on a threshold which has to be set first, a parking space will be classified as either vacant or occupied depending on the object count.

Another common image processing technique is to use foreground/background information of the images. Using this approach the main background color is being identified and compared to the whole image. The background of a parking space can either be defined via extracting a certain part of an image which should always represent the main background color of a parking space's pavement (done by Blumer et al. \cite{Blumer2012}) or via an histogram of the image assuming that the background uses the most pixels in a recorded image (done by Liu et al. \cite{stationary_camera_sensing}). After the background color is available it is being subtracted from the original image and using thresholding foreground and background pixels are identified and counted. Depending on the count of the respective pixels, the parking space is then classified as vacant or occupied.

Liu et al. \cite{stationary_camera_sensing} used all of the above mentioned techniques to build a more stable prototype. They only tested their prototype indoors which is why weather and lightning conditions were no problems. With sensing only a maximum of seven cars, it provided an ensemble technique which should be more reliable. However, they did not include a statistic of how well their algorithm performed on a bigger dataset. Another ensemble method was developed by Blumer \cite{Blumer2012}. They used edge counts and background/foreground information as input for different machine learning techniques, which should then classify a parking space as vacant or occupied. Their plain algorithm achieved an accuracy of about 77.8\%. However, they improved it using frames of preceding and following images to identify parking/unparking events and then achieved an accuracy of about 88.8\%.

%\cite{Marmol2016}  Streetline, 
%arm smart camera \cite{stationary_camera_sensing_arm_smart_camera}

\paragraph{Parking Detection using Deep Learning and CNNs}

In recent years deep learning, which is a specific variant of neural networks, gained a lot of significance in the field of machine learning. For example in image recognition, convolutional neural networks (CNNs) preform nowadays as well as humans in classifying everyday objects in digital images. A CNN is a neural network with a possibly large amount of hidden layers of which some of them are convolutional layers, which take into account the spatial relationship between neighbouring pixels (in the case of image recognition). This rise in learning power also leads to projects which try to train CNNs in classifying the states of parking spaces while using camera images of a parking area.

In 2016 Amato et al. \cite{Amato2016} trained two different CNNs on classifying parking space states and compared their performances afterwards. They used the publicly available PKLot dataset as well as a dataset collected by themselves to train and evaluate their neural networks. In total they had over 700.000 images in different weather conditions and with different levels of quality (in some situations parking lots were almost fully occluded by trees or lamp posts, etc.). Results of the CNN's classifications were promising. Despite the fact that some of the images did not exactly match the parking space and there were a lot of occlusions, the best performing deep neural network gained an accuracy of above 91\% on all subsets of the datasets. 

Another work in the field of deep learning has been done by Di Mauro also in 2016 \cite{DiMauro2016}. They also used CNNs and the PKLot dataset as well as a self acquired dataset. However, they had a slightly different CNN as they also used a technique called pseudo-labelling of the data which can be seen as semi-supervised approach. When using pseudo-labelling both labelled and unlabelled data are used at the same time to train the network. For unlabelled data the label which was computed by the CNN in the forward pass is used, that is also why it is called pseudo-labelling. Furthermore, a different loss function has to be used as the pseudo labelled data has much less significance than the ground truth labelled data. Di Mauro et al. trained the CNN with about 5\% of the data and achieved an accuracy of above 96\% with a the fully supervised approach. The pseudo-labelling approach reached about the same levels of accuracy, however the dataset had to be balanced for it to work properly.


\paragraph{Advantages and Disadvantages of using stationary Cameras}

As described in this section, there is already a lot of research how cameras can be used to classify the states of parking spaces. Advantages of cameras are the high level of accuracy they reach while covering a lot of cars at once. However, for all of the above discussed approaches to work cameras would have to manually calibrated before usage to know where parking spaces are. Furthermore, most of the research focused on the use of cameras for detection the parking space situation of a single parking lot with a lot of cars rather than for road side parking spaces. For this approach to work at a city wide scope, cameras would have to be mounted at least at every block. Similar to the dedicated sensors per parking space (section \ref{sec:stationary_park_sensing}) this would cause high costs not only in the form of hardware costs but also in installation and maintenance. Furthermore, mounting cameras throughout the city obviously brings up privacy issues.





\subsection{Counting In- and Outgoing Vehicles}
\label{sec:counting_in_out_park_sensing}

A rather simple approach of estimating the number of free parking spaces is to count in- and outgoing vehicles at parking zones. Zadeh et al. \cite{smarturbanparkingdetection} developed a prototype which counts all in- and outgoing vehicles using a Raspberry Pi and two ultrasonic range finders at each gate. Ultrasonic sensors measure the distance from the side of the road to a potentially passing vehicle. Two ultrasonic sensors are needed to detect if a vehicle is going in or out. According to the order in which the ultrasonic sensors detect a passing vehicle it is decided whether a vehicle is going in or out. Furthermore, the two sensors are also used to differentiate a passing person from a vehicle. Both sensors are as far apart as they can detect a vehicle at the same time but not a person. Using this technique misleading detections of passengers are filtered.

As described this technique is technically simple to develop and the hardware costs are also quite cheap. However, this system is only designed for parking areas which have a low number of gates where cars go in and out and not for a city wide deployment. For closed parking areas it is easy to detect in- and outgoing vehicles whereas it is hard to do so in open traffic and street scenarios. Furthermore, the counting also will not work with ultrasonic sensors in open street scenarios as there are often several lanes, which would make the sensing impossible as passing cars would possibly occlude each other.






\subsection{Event Detection based Park Sensing using Smartphones}
\label{sec:event_detection_park_sensing}

Detecting parking and unparking events using a driver's smartphone is another approach to estimate parking space availabilities, which is especially interesting because nowadays almost everybody has a personal smartphone. As soon as these parking related information is available for a high enough rate of drivers, a city's parking availability situation can be roughly estimated. There already exist a few research papers which discuss different approaches to this topic. Three of these will be discussed in this section.

In 2013 Nawaz et al. developed ParkSense \cite{Nawaz:2013:PSB:2500423.2500438}, which is an application for Android phones to detect certain events which relate to the change of parking situations. Using the app a user could log his parking events and furthermore the costs of parking. The app was released to the Google Play Store and recorded 59 parking traces in four different cities at the time their paper was released. However, the app not only recorded the times when parking and unparking took place, but also recorded a detail profile of the Wifi access points while parking as well as Wifi profiles and GPS locations after unparking took place. Using this dataset, the goal was to evaluate how accurately unparking events could be detected using Wifi fingerprints. The approach of Nawaz et al. was to take several Wifi fingerprints when parking takes place and then continuously scan the Wifi signal and compare it to the signal while parking. Using this technique ParkSense could detect when users returned to their vehicle. However, it cannot be assumed that when a user returns close to his vehicle, that he will leave with his car, therefore Nawaz et al. also implemented an activity detection for driving using the change of Wifi signals over a specific timespan. The idea behind this approach is, that while driving, Wifi access points in range will change much more frequently than while walking or while staying at the same place. Out of 41 cases where unparking actually took place ParkSense could detected 38 user returns correctly and 35 times out of the 41 it could detect that the vehicle started driving. Thus, ParkSense reached an overall accuracy of about 85\% in detecting unparking events.

A similar approach to the problem of parking related event detection has been done by Ma et al. \cite{Ma:2014:USP:2674918.2674929} in 2014. The overall goal of this research paper was to robustly identify parking and unparking events using different sensors of a typical smartphone. They identified in total nine indicators which could be sensed whether a parking/unparking event occurred. For instance, they used several accelerometer measurements to identify the change of the user from walking to driving and vice versa. Other indicators used were Bluetooth signal of the car, sounds of a started motor sensed via a smartphones microphone, Wifi fingerprints and accessing parking payment apps. To decrease the amount of energy consumed by the application, certain low power sensors (e.g. accelerometer) measured periodically while other high consumption sensors (e.g. microphone) only confirmed or rejected events which were already sensed by low power sensors. Using all indicators, Ma et al. built a model based on conditional probabilities which should be able to show the probability of an event at a certain time. To evaluate their approach, they developed an Android application which they used to manually collect their ground truth events. They then used 40 parking samples to train their model and tested it on 60 other samples. Their best configuration achieved 93\% recall and 90\% precision on parking events and 81\% recall and 93\% precision on unparking events.

\paragraph{Estimating Events by non-tracked Drivers}

All of the above mentioned approaches would only work if almost all drivers in a specific city would have had the respective smartphone app installed, so that all parking and unparking events could be detected. That way the states of all parking spaces would be known and parking availability could be derived. However, this is highly unlikely as people would have to know that such an app exists in the first place and furthermore there is no instant positive effect of having such an application installed for a single user. Furthermore, such an app might also have a negative influence on the battery life of a user's smartphone even if all of the above applications are designed for a low power consumption. 

Facing this problem in 2014 Nandugudi et al. proposed PocketParker \cite{Nandugudi:2014:PPP:2632048.2632098}, a smartphone application which should not only detect parking and unparking events but also estimate the amount of so called "hidden drivers" - drivers that do not have the application installed. PocketParker uses a simple model to detect parking and unparking events. It uses the Google Play Services activity recognition library to detect changes from driving to walking and vice versa and thus derive parking and unparking events. After this information is available, PocketParker tries to estimate the amount of drivers having the application installed, so that all detected events could be scaled to the right proportion and taken into the overall parking estimation process. Nangundi et al. not only identify the parking space location, but only the user's end target after parking his vehicle. That way they want to identify parking spaces which would be nearer to the end destination of the user, but probably are full since the user parked further away and did not took the closest parking space. Using this approach PocketParker should derive parking space counts at dedicated parking lots which parking space count was known before. To evaluate its performance Nangundi et al. performed an experiment at their university's parking lot. 105 users generated 10.827 events over 45 days, so an average of 241 events per day. They used cameras to identify the ground truth and then evaluated their results against it. With an estimated driver fraction (drivers which had the app installed) of about 20\%, they achieved a parking space count accuracy of about 94\% for the university's parking lot.

\paragraph{Advantages and Disadvantages of using an event-based Approach}

There are obviously a number of advantages using smartphones to detect parking availabilities. Nowadays almost everybody has at least one smartphone which he/she carries around all day. Therefore smartphones are the optimal sensing devices for detecting a user's activities throughout the day. Also parking and unparking activities can be detected and parking availabilities can be derived. The whole detection process would cause no additional costs for hardware installations and maintenance for city authorities. Furhtermore, the detection process could all be done without the user even noticing, because the detection process could theoretically run fully as a background process. 

However, a big concern of mobile applications is power usage. Activity detection requires certain sensors to run periodically which prevents the phone from being idle and therefore uses more battery. Even if all of the discussed approaches intentionally use high power consuming sensors like GPS sensors or microphones only in rare cases, the immanent sensing of low power consuming sensors will also cause decreased battery hours. Another disadvantage is that a lot of users would have to install the app for accurate parking availability estimations. Even if Nangundi et al. \cite{Nandugudi:2014:PPP:2632048.2632098} showed that the so called "hidden drivers" can be estimated to a good degree, it probably will be hard to get even a user base of 20\% as they proposed in their paper. The first obstacle would be to let users even know that there exists such an app, so that they can install it. The following problem would be the cold start problem. After the app is released, usually only a few people will have it installed and they will probably not see an advantage of such an app because the availability estimates will not be accurate if there is only a really small user base. 






\subsection{Drive-by Park Sensing using Cameras}
\label{sec:related_driveby_park_sensing_cameras}
 
In contrast to approaches with stationary sensors which were discussed in sections \ref{sec:stationary_park_sensing} and \ref{sec:stationary_park_sensing_cameras}, drive-by sensing approaches use sensors which are mounted on vehicles driving through the city to sense vacant and occupied parking spaces. Using such mobile sensors has the goal of achieving a high enough accuracy at a fraction of the costs of stationary solutions. In this section the usage of cameras mounted on driving vehicles to detect parking spaces is discussed.

There exist a lot of different approaches in detecting a parking space from a picture of a vehicle mounted camera. Most of them were designed for the purpose of park assistance, so that parking is being made easier for drivers or that the vehicle can even park itself. However, almost all of such approaches can also be applied for camera-based drive by sensing. 

The first discussed approach is to use parking markings. Xu et al. \cite{Xu2000} implemented a color vision based parking space detection algorithm which tried to identify parking markings on the street. They tried to use neural networks to learn to distinguish parking marking pixels from others and they used stereo vision to identify obstacles on parking spaces. A similar approach has been done by Jung et al. \cite{Jung2006} who implemented a hough transform on the captured images to detect lines (parking markings) on the images. However, the accuracy of color vision based approaches, as just discussed can vary highly in different lightning conditions, conclusions or shadows. Furthermore, of course this approach only works with parking spaces where markings are present.

Another approach is to use two or more images to create depth maps to estimate vacant parking spaces. Kaempchen et al. \cite{Kaempchen2002} developed a stereo camera based approach which estimated depth maps based on two images of calibrated cameras placed on the vehicle next to each other. By finding features in both images and correlating these, the disparity and thus the depth of points can be estimated and vehicles as well as free spaces can be detected using the 3D information. However, a stereo camera solution has the drawbacks of relying on sensitively calibrated cameras which should have a reasonable resolution to be able to match the features in both images. 

In 2010 Suhr et al. \cite{Suhr2010} developed an approach using a fish eye camera mounted on a vehicle, which is facing backwards.
 
parkmaster \cite{Grassi:2017:PIE:3132211.3134452}



\subsection{Drive-by Park Sensing using Distance Sensors}
\label{sec:related_driveby_park_sensing_distance}

\subsubsection{ParkNet}
\label{sec:parknet}

parknet... \cite{Mathur:2010:PDS:1814433.1814448}



\section{Acquiring Parking Space Maps}



\section{Machine Learning}

