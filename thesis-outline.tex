%%% Header
\documentclass[11pt,a4paper,titlepage,
chapterprefix,headsepline,parskip,pdftex,
,pointlessnumbers,bibtotoc]{article}

%%% paragraph for lower depth
\makeatletter %% remove @ meaning
\renewcommand{\paragraph}{\@startsection
   {paragraph} % name
   {4} % depth
   {0mm} % indention
   {-\baselineskip} % before
   {0.1\baselineskip} % after
   {\normalfont\normalsize\bfseries}} % stil
\makeatother %% add @ meaning

\makeatletter %% remove @ meaning
\renewcommand{\subparagraph}{\@startsection
   {subparagraph} % name
   {5} % depth
   {0mm} % indention
   {-\baselineskip} % before
   {0.1\baselineskip} % after
   {\normalfont\normalsize\bfseries}} % stil
\makeatother %% add @ meaning

\usepackage{setspace}
\onehalfspacing

\usepackage[pdftex]{graphicx}

% for colours
\usepackage[pdftex]{color}

% Configure defintions
\usepackage{amsthm}
\newtheorem{mydef}{Definition}

\usepackage{tikz}
\usetikzlibrary{shapes,arrows}
% \usepackage{subfigure} // cannot be used with subcaption
\usepackage{caption}
\usepackage{subcaption}

\usepackage[colorlinks=true,
    linkcolor=black,
    citecolor=black,
    pagecolor=black,
    urlcolor=black,
    breaklinks=true,
    bookmarksnumbered=true,
    hypertexnames=false,
    pdfpagemode=UseOutlines,
    pdfview=FitH,
    plainpages=false,
    pdfpagelabels,
    bookmarks=true,
    linktocpage=true]{hyperref}

\hypersetup{pdfauthor={Markus Hiesmair},
    pdftitle={Austrian Parliament Analyzer},
    pdfsubject={Bachelor's Thesis},
    pdfkeywords={},
    pdfcreator={pdfLaTeX with hyperref (\today})}

%%% Source-Code
\usepackage{listings}
\lstset{language=Java}

\lstset{% general command to set parameter(s)
basicstyle=\small, % print whole listing small
tabsize=2, %
keywordstyle=\color[rgb]{0.00,0.00,0.50}{}\bfseries,
%identifierstyle=, % nothing happens
commentstyle=\color[rgb]{0.00,0.50,0.25}{},
%stringstyle=\ttfamily, % typewriter type for strings
%showstringspaces=false} % no special string spaces
numbers=left, numberstyle=\tiny, numbersep=5pt}

\colorlet{punct}{red!60!black}
\definecolor{background}{HTML}{EEEEEE}
\definecolor{delim}{RGB}{20,105,176}
\colorlet{numb}{magenta!60!black}

% Custom definition for JSON
\lstdefinelanguage{json}{
    basicstyle=\normalfont\ttfamily,
    numbers=left,
    numberstyle=\scriptsize,
    stepnumber=1,
    numbersep=8pt,
    showstringspaces=false,
    breaklines=true,
    frame=lines,
    backgroundcolor=\color{background},
    literate=
     *{0}{{{\color{numb}0}}}{1}
      {1}{{{\color{numb}1}}}{1}
      {2}{{{\color{numb}2}}}{1}
      {3}{{{\color{numb}3}}}{1}
      {4}{{{\color{numb}4}}}{1}
      {5}{{{\color{numb}5}}}{1}
      {6}{{{\color{numb}6}}}{1}
      {7}{{{\color{numb}7}}}{1}
      {8}{{{\color{numb}8}}}{1}
      {9}{{{\color{numb}9}}}{1}
      {:}{{{\color{punct}{:}}}}{1}
      {,}{{{\color{punct}{,}}}}{1}
      {\{}{{{\color{delim}{\{}}}}{1}
      {\}}{{{\color{delim}{\}}}}}{1}
      {[}{{{\color{delim}{[}}}}{1}
      {]}{{{\color{delim}{]}}}}{1},
}

\lstset{
    basicstyle=\normalfont\ttfamily,
    numbers=left,
    numberstyle=\scriptsize,
    stepnumber=1,
    numbersep=8pt,
    showstringspaces=false,
    breaklines=true,
    frame=lines,
    backgroundcolor=\color{background},
}


% Figure
\newcommand{\cffigure}[1]{\hyperref[#1]
{cf. \ref*{#1}~\nameref{#1}}}


%%% continous footnote
\newcounter{cfootnotecounter}
\newcommand{\cfootnote}[1]{\stepcounter{cfootnotecounter}
\footnote[\value{cfootnotecounter}]{#1}}

\usepackage{footnote}
\makesavenoteenv{tabular}
\makesavenoteenv{table}

\flushbottom

% change page settings
\setlength{\hoffset}{0mm} \setlength{\voffset}{0mm}
\setlength{\evensidemargin}{14.6mm}
\setlength{\oddsidemargin}{14.6mm} \setlength{\topmargin}{-20mm}
\setlength{\headheight}{15mm} \setlength{\headsep}{9mm}
\setlength{\textheight}{242mm} \setlength{\textwidth}{145mm}
\setlength{\footskip}{10mm}


%%% seperation of float-environment
\setlength{\textfloatsep}{25pt plus5pt minus5pt}
\setlength{\intextsep}{25pt plus5pt minus5pt}

%%% Gliederungs-Nummern in den Rand schreiben
%\renewcommand*{\othersectionlevelsformat}[1]{%
%\llap{\csname the#1\endcsname\autodot\enskip}}

%%% only chapter name in headline
%\renewcommand*{\chaptermarkformat}{}

%%% format of the chapter looks
%\setkomafont{chapter}{\Huge}
%\renewcommand*{\chapterformat}{\LARGE{\chapappifchapterprefix{\ }\thechapter\autodot\enskip}}

%%% headline
\usepackage[automark]{scrpage2}

\clearscrheadings \clearscrplain \clearscrheadfoot
\pagestyle{scrheadings}
\ohead{\pagemark}
\ihead{\headmark}
\cfoot{}

%%% design of chapter pages
%\renewcommand*{\chapterpagestyle}{scrheadings}

%% indices from TOC and numbering depth
\setcounter{tocdepth}{\subsubsectionlevel}
\setcounter{secnumdepth}{\paragraphlevel}

%%% Array for tables
\usepackage{array}

%%% fonts
%\addtokomafont{chapter}{\sffamily}
%\addtokomafont{sectioning}{\rmfamily}

% language
\usepackage[english]{babel}
% inputencoding
\usepackage[utf8]{inputenc}
% fontencoding
\usepackage[T1]{fontenc}
\usepackage{ae}

% URLs
\usepackage{url}

%%% orphan  und widow
\clubpenalty = 10000
\widowpenalty = 10000 \displaywidowpenalty = 10000


%%% Einbinden von kompletten PDF-Seiten
\usepackage{pdfpages}

\usepackage{subcaption}
\expandafter\def\csname ver@subfig.sty\endcsname{}

\usepackage{ifluatex}
\ifluatex
  \usepackage{pdftexcmds}
  \makeatletter
  \let\pdfstrcmp\pdf@strcmp
  \let\pdffilemoddate\pdf@filemoddate
  \makeatother
\fi

\usepackage{svg}
\usepackage{amsmath}
\usepackage{xr}
\usepackage{epstopdf}
\usepackage{wrapfig}

% comment the following line for final submission
%\newcommand{\manuscriptmode}[0]{draft}

\newcommand{\thesistitle}[0]{Making Computers Understand Coalition and Opposition in Parliamentary Democracy}

%%% Custom commands
% provides support for a \todo command
\input{customc.tex}

%%% hyphenation
\input{structure/hyphenation}

\begin{document}

%%% cover page
\pagenumbering{roman}
\includepdf{structure/cover/cover.pdf}

%%% structure parts
\pagenumbering{Roman}
%\ifx\manuscriptmode\undefined{}
	%\include{structure/dedication}
%\fi
%\include{structure/affidavit}
%\chapter*{Acknowledgment}
\markright{Acknowledgment}

\ifx\manuscriptmode\undefined{}

Hereby I would like to thank my supervisor Dr. Karin Anna Hummel for her support throughout the whole work on this thesis. She always provided me useful feedback and used her professional skills and knowledge about this topic to give me hints and advices when I needed it. 

\fi

%%\chapter*{Summary}
%\markright{Summary}

%Summary \ldots

\chapter*{Abstract}
\markright{Abstract}

Traffic congestion is one of the largest problems in urban areas nowadays. Not only the increased number of daily trips contributes to traffic jams, but also drivers who are looking for parking spaces. When driving around a long time to find vacant parking spaces, vehicles not only congest the streets, but also emit a significant amount of green house gases. To mitigate these problems, increased transparency of the current parking space availability situation is desirable to help drivers navigate directly to vacant parking spaces close to their end destination. 

This thesis tackles the task of creating parking space availability maps for cities to make it easier for drivers to find vacant parking spaces. A drive-by approach is chosen to detect vacant and occupied parking spaces. 
Using this approach, several vehicles are driving through the city and are continuously collecting sensor data. All vehicles measure the distance to the right side of the road as well as their position using GPS. Using the distance information, our prototype should be able to decide whether a parking space is occupied or vacant. Furthermore, distracting situations which prevent a successful detection (e.g. an overtaking situation) should be detected as well. 
A prototype vehicle has been equipped with the necessary sensors. Real world sensor data has been collected during 32 test drives and is used in the evaluation process to identify the best performing systems.

As the traffic situations in our test drives can be fairly complicated, it is non-trivial to find a rule set which can detect the different situations. Thus, machine learning techniques are tested on their performance. Using a manually tagged dataset, containing nine computed features derived from the sensor data, several traditional machine learning algorithms and deep learning approaches are evaluated on their classification performance. Furthermore, several approaches to further improve the classification performance are tested, such as over- and under-sampling. 
Finally, a custom two-staged classification process is introduced to include information of the surroundings of the parking situation to classify. This approaches reaches the best classification results. In total, 96.52\% of the situations are correctly classified and 93.81\% of the parking cars are detected.



%%%% Diverse Verzeichnisse

\setcounter{secnumdepth}{5}
\setcounter{tocdepth}{5}
\tableofcontents
\todo{For now the TOC depth is 5 but will be reduced later}


\chapter*{Abbreviations}
\markright{Abbreviations}

\begin{description}
\setlength{\itemsep}{-11pt}
\setlength{\leftmargin}{900pt}

% C
\item[CAB] Compute Aggregate Broadcast (A computing model in parallel computing where computation is strictly partitioned in the three phases compute, aggregate and broadcast)

% I
\item[IaaS] Infrastructure as a Service (Cloud computing service layer)

% J
\item[JSF] Java Server Faces (Web technology in the arena of Java enterprise)
\item[JSP] Java Server Pages (Web technology available in Java Servlet containers)

% M
\item[MPI] Message Passing Interface (Standard for implementing parallel algorithms on shared-nothing infrastructures)

% O
\item[OSN] Online Social Network (An usually web-based online platform where friends, and acquaintances can connect and share information)

% P
\item[PaaS] Platform as a Service (Cloud computing service layer)

% S
\item[SaaS] Software as a Service (Cloud computing service layer)

% U
\item[UML] Unified Modeling Language
\item[URL] Uniform Resource Locator

% V
\item[VM] Virtual Machine

% W
\item[WAR] Web Application Archive

\end{description}


\listoffigures

\listoftables


%%% content
\newpage
\pagenumbering{arabic}

\section{Initial Situation and Motivation}
Currently the road side parking situation in most cities is rather untransparent. Except from parking garages and the like information about the availability of parking spaces is rarely available. However, such information can help to reduce traffic by a tremendous amount. Studies have shown that in urban areas about 30\% of traffic congestion is created by drivers looking for free parking spaces \cite{Nawaz:2013:PSB:2500423.2500438} and that in 2007 a loss of about \$78 billion U.S. dollars was created by the use of about 2.9 billion gallons of gasoline alone in the USA \cite{TexasMobilityReport}. Furthermore, about 4.9 billion hours were wasted by drivers while looking for parking spaces during that year. 

\section{Problem Definition}

Detection of road side parking spaces and their states is a challenging task. Of course an obvious approach to the problem would be to put sensors to every parking space in the city, which check, if the corresponding parking space is occupied or vacant. This, however, has the drawback to be very expensive as, for big cities, thousands of sensors would have to be bought, installed and maintained. Furthermore, because the parking situation does not change often, the high frequency of sensing with such a system would be rather inefficient.

Another approach to sensing a city's parking situation is the use of mobile sensors instead of static ones.
For instance, vehicles which drive through the cities could sense parking spaces while they drive by. There already exists some work on this subject. Mathur et al. \cite{Mathur:2010:PDS:1814433.1814448} developed their system "ParkNet" which can sense parking spaces using an ultrasonic range finder. 
While the car is driving through the city, the sensor continuously measures the distance to the nearest obstacle on the right side of the road (in many cases a parking car). Using this information and GPS measurements the parking space counts and the parking occupancy rates could be derived by an accuracy of over 90\%.

However, such mobile sensing systems currently only work on single lane roads. To work in real world scenarios, lane detection has to be incorporated. 



\section{Goals and Detailed Approach}


%In this thesis, the parking situation should be sensed differently. Vehicles should be used as mobile sensors, and while they drive through the city, the collected sensor measurements should be used to derive the current parking availability situation. Such a system should have the advantage to be more cost effective and should also provide a sufficient accurate estimate of the current parking availability in real time.

\section{Milestones}

Table \ref{tab:milestones} shows the planned milestones of this project.

\begin{table}[h]
\centering
\begin{tabular}{r|l}
\emph{Date} & \emph{Milestone} \\\hline
07.04.2017 & Hardware is available \\
28.04.2017 & Hardware parts work together and sensor data can be retrieved. \\
 & Tests regarding the accuracy and range of the optical distance sensor are taken. \\
12.05.2017 & The sensors are mounted on the car. \\
19.05.2017 & Test data has been collected on a single lane road. \\
31.06.2017 & Single lange parking detection is implemented and has been evaluated. \\
15.07.2016 & Test data has been collected on multi lane roads in different scenarios. \\
01.09.2016 & Lane detection algorithm has been implemented and evaluated. \\
31.10.2017 & Parking detection on multi lane road has been implemented and evaluated \\
01.12.2017 & Submission of the thesis
\end{tabular}
\caption{\label{tab:milestones} Milestones}
\end{table}

%%% appendix
%\appendix

%\chapter{Machine Learning Results}
\label{appendix:ml_results}
\markright{Machine Learning Experiment Results}

This appendix lists the results of the best configuration of all discussed traditional machine learning models in the form of confusion matrices. For each model, a table with the results for the full- and filtered dataset is provided.


% random forest
\begin{table}[h]

\resizebox{\textwidth}{!}{%
\centering
\bgroup
\def\arraystretch{1.4}
\begin{tabular}{| r || c | c | c | c |}
\hline
   Predicted class $\rightarrow$ &
   \textbf{Free Space} & 
   \textbf{Parking Car} &
   \textbf{Overtaking} &
   \textbf{Other Parking} \\
   True class $\downarrow$ &
	 & 
	 &
	 \textbf{Situation} &
	 \textbf{Vehicle} \\
\hline
  \textbf{Free Space} & 11718 & 90 & 0 & 1 \\
\hline
  \textbf{Parking Car} & 244 & 1950 & 21 & 1 \\
\hline
  \textbf{Overtaking Situation} & 63 & 82 & 62 & 0 \\
\hline
  \textbf{Other Parking Vehicle} & 54 & 14 & 0 & 1 \\
\hline

\end{tabular}
\egroup
}

\begin{center}
(a) full dataset\\
\end{center}

\resizebox{\textwidth}{!}{%
\centering
\bgroup
\def\arraystretch{1.4}
\begin{tabular}{| r || c | c | c | c |}
\hline
   Predicted class $\rightarrow$ &
   \textbf{Free Space} & 
   \textbf{Parking Car} &
   \textbf{Overtaking} &
   \textbf{Other Parking} \\
   True class $\downarrow$ &
	 & 
	 &
	 \textbf{Situation} &
	 \textbf{Vehicle} \\
\hline
  \textbf{Free Space} & 6598 & 81 & 0 & 1 \\
\hline
  \textbf{Parking Car} & 164 & 2017 & 2 & 0 \\
\hline
  \textbf{Overtaking Situation} & 14 & 35 & 17 & 0 \\
\hline
  \textbf{Other Parking Vehicle} & 37 & 18 & 0 & 5 \\
\hline

\end{tabular}
\egroup
}

\begin{center}
(b) filtered dataset\\
\end{center}

\caption{Confusion matrices of a random forest classifier (containing 1000 trees and using entropy as criterion for node-splitting) applied on (a) the full dataset and (b) the filtered dataset.}
\label{table:cf_rf}
\end{table}





% decision tree
\begin{table}

\resizebox{\textwidth}{!}{%
\centering
\bgroup
\def\arraystretch{1.4}
\begin{tabular}{| r || c | c | c | c |}
\hline
   Predicted class $\rightarrow$ &
   \textbf{Free Space} & 
   \textbf{Parking Car} &
   \textbf{Overtaking} &
   \textbf{Other Parking} \\
   True class $\downarrow$ &
	 & 
	 &
	 \textbf{Situation} &
	 \textbf{Vehicle} \\
\hline
  \textbf{Free Space} & 11616 & 171 & 15 & 9 \\
\hline
  \textbf{Parking Car} & 283 & 1882 & 50 & 1 \\
\hline
  \textbf{Overtaking Situation} & 59 & 83 & 65 & 0 \\
\hline
  \textbf{Other Parking Vehicle} & 49 & 16 & 0 & 4 \\
\hline

\end{tabular}
\egroup
}

\begin{center}
(a) full dataset\\
\end{center}

\resizebox{\textwidth}{!}{%
\centering
\bgroup
\def\arraystretch{1.4}
\begin{tabular}{| r || c | c | c | c |}
\hline
   Predicted class $\rightarrow$ &
   \textbf{Free Space} & 
   \textbf{Parking Car} &
   \textbf{Overtaking} &
   \textbf{Other Parking} \\
   True class $\downarrow$ &
	 & 
	 &
	 \textbf{Situation} &
	 \textbf{Vehicle} \\
\hline
  \textbf{Free Space} & 6541 & 125 & 1 & 13 \\
\hline
  \textbf{Parking Car} & 200 & 1962 & 13 & 8 \\
\hline
  \textbf{Overtaking Situation} & 14 & 37 & 15 & 0 \\
\hline
  \textbf{Other Parking Vehicle} & 32 & 19 & 0 & 9 \\
\hline

\end{tabular}
\egroup
}

\begin{center}
(b) filtered dataset\\
\end{center}

\caption{Confusion matrices of a decision tree (using gini impurity as criterion for node-splitting) applied on (a) the full dataset and (b) the filtered dataset.}
\label{table:cf_tree}
\end{table}






% kNN
\begin{table}

\resizebox{\textwidth}{!}{%
\centering
\bgroup
\def\arraystretch{1.4}
\begin{tabular}{| r || c | c | c | c |}
\hline
   Predicted class $\rightarrow$ &
   \textbf{Free Space} & 
   \textbf{Parking Car} &
   \textbf{Overtaking} &
   \textbf{Other Parking} \\
   True class $\downarrow$ &
	 & 
	 &
	 \textbf{Situation} &
	 \textbf{Vehicle} \\
\hline
  \textbf{Free Space} & 11581 & 231 & 14 & 3 \\
\hline
  \textbf{Parking Car} & 276 & 1915 & 24 & 1 \\
\hline
  \textbf{Overtaking Situation} & 97 & 80 & 30 & 0 \\
\hline
  \textbf{Other Parking Vehicle} & 51 & 14 & 0 & 4 \\
\hline

\end{tabular}
\egroup
}

\begin{center}
(a) full dataset\\
\end{center}

\resizebox{\textwidth}{!}{%
\centering
\bgroup
\def\arraystretch{1.4}
\begin{tabular}{| r || c | c | c | c |}
\hline
   Predicted class $\rightarrow$ &
   \textbf{Free Space} & 
   \textbf{Parking Car} &
   \textbf{Overtaking} &
   \textbf{Other Parking} \\
   True class $\downarrow$ &
	 & 
	 &
	 \textbf{Situation} &
	 \textbf{Vehicle} \\
\hline
  \textbf{Free Space} & 6491 & 181 & 1 & 7 \\
\hline
  \textbf{Parking Car} & 177 & 2003 & 3 & 0 \\
\hline
  \textbf{Overtaking Situation} & 28 & 37 & 1 & 0 \\
\hline
  \textbf{Other Parking Vehicle} & 35 & 19 & 0 & 6 \\
\hline

\end{tabular}
\egroup
}

\begin{center}
(b) filtered dataset\\
\end{center}

\caption{Confusion matrices of a kNN classifier (using the five nearest neighbours) applied on (a) the full dataset and (b) the filtered dataset.}
\label{table:cf_knn}
\end{table}





% neural network
\begin{table}

\resizebox{\textwidth}{!}{%
\centering
\bgroup
\def\arraystretch{1.4}
\begin{tabular}{| r || c | c | c | c |}
\hline
   Predicted class $\rightarrow$ &
   \textbf{Free Space} & 
   \textbf{Parking Car} &
   \textbf{Overtaking} &
   \textbf{Other Parking} \\
   True class $\downarrow$ &
	 & 
	 &
	 \textbf{Situation} &
	 \textbf{Vehicle} \\
\hline
  \textbf{Free Space} & 11587 & 214 & 9 & 1 \\
\hline
  \textbf{Parking Car} & 319 & 1879 & 17 & 1 \\
\hline
  \textbf{Overtaking Situation} & 78 & 92 & 37 & 0 \\
\hline
  \textbf{Other Parking Vehicle} & 51 & 18 & 0 & 0 \\
\hline

\end{tabular}
\egroup
}

\begin{center}
(a) full dataset\\
\end{center}

\resizebox{\textwidth}{!}{%
\centering
\bgroup
\def\arraystretch{1.4}
\begin{tabular}{| r || c | c | c | c |}
\hline
   Predicted class $\rightarrow$ &
   \textbf{Free Space} & 
   \textbf{Parking Car} &
   \textbf{Overtaking} &
   \textbf{Other Parking} \\
   True class $\downarrow$ &
	 & 
	 &
	 \textbf{Situation} &
	 \textbf{Vehicle} \\
\hline
  \textbf{Free Space} & 6493 & 181 & 6 & 0 \\
\hline
  \textbf{Parking Car} & 214 & 1966 & 3 & 0 \\
\hline
  \textbf{Overtaking Situation} & 21 & 40 & 5 & 0 \\
\hline
  \textbf{Other Parking Vehicle} & 36 & 24 & 0 & 0 \\
\hline

\end{tabular}
\egroup
}

\begin{center}
(b) filtered dataset\\
\end{center}

\caption{Confusion matrices of a neural network (using five layers with each 50 hidden units, trained during 1.000.000 epochs) applied on (a) the full dataset and (b) the filtered dataset.}
\label{table:cb_nn}
\end{table}



% svm
\begin{table}

\resizebox{\textwidth}{!}{%
\centering
\bgroup
\def\arraystretch{1.4}
\begin{tabular}{| r || c | c | c | c |}
\hline
   Predicted class $\rightarrow$ &
   \textbf{Free Space} & 
   \textbf{Parking Car} &
   \textbf{Overtaking} &
   \textbf{Other Parking} \\
   True class $\downarrow$ &
	 & 
	 &
	 \textbf{Situation} &
	 \textbf{Vehicle} \\
\hline
  \textbf{Free Space} & 11717 & 93 & 1 & 0 \\
\hline
  \textbf{Parking Car} & 453 & 1762 & 1 & 0 \\
\hline
  \textbf{Overtaking Situation} & 172 & 35 & 0 & 0 \\
\hline
  \textbf{Other Parking Vehicle} & 56 & 13 & 0 & 0 \\
\hline

\end{tabular}
\egroup
}

\begin{center}
(a) full dataset\\
\end{center}

\resizebox{\textwidth}{!}{%
\centering
\bgroup
\def\arraystretch{1.4}
\begin{tabular}{| r || c | c | c | c |}
\hline
   Predicted class $\rightarrow$ &
   \textbf{Free Space} & 
   \textbf{Parking Car} &
   \textbf{Overtaking} &
   \textbf{Other Parking} \\
   True class $\downarrow$ &
	 & 
	 &
	 \textbf{Situation} &
	 \textbf{Vehicle} \\
\hline
  \textbf{Free Space} & 6607 & 73 & 0 & 0 \\
\hline
  \textbf{Parking Car} & 377 & 1806 & 0 & 0 \\
\hline
  \textbf{Overtaking Situation} & 52 & 14 & 0 & 0 \\
\hline
  \textbf{Other Parking Vehicle} & 42 & 18 & 0 & 0 \\
\hline

\end{tabular}
\egroup
}

\begin{center}
(b) filtered dataset\\
\end{center}

\caption{Confusion matrices of a support vector machine applied on (a) the full dataset and (b) the filtered dataset.}
\label{table:cf_svm}
\end{table}




% naive bayes
\begin{table}

\resizebox{\textwidth}{!}{%
\centering
\bgroup
\def\arraystretch{1.4}
\begin{tabular}{| r || c | c | c | c |}
\hline
   Predicted class $\rightarrow$ &
   \textbf{Free Space} & 
   \textbf{Parking Car} &
   \textbf{Overtaking} &
   \textbf{Other Parking} \\
   True class $\downarrow$ &
	 & 
	 &
	 \textbf{Situation} &
	 \textbf{Vehicle} \\
\hline
  \textbf{Free Space} & 5013 & 2159 & 561 & 4078 \\
\hline
  \textbf{Parking Car} & 55 & 1942 & 78 & 141 \\
\hline
  \textbf{Overtaking Situation} & 9 & 89 & 79 & 30 \\
\hline
  \textbf{Other Parking Vehicle} & 1 & 11 & 0 & 57 \\
\hline

\end{tabular}
\egroup
}

\begin{center}
(a) full dataset\\
\end{center}

\resizebox{\textwidth}{!}{%
\centering
\bgroup
\def\arraystretch{1.4}
\begin{tabular}{| r || c | c | c | c |}
\hline
   Predicted class $\rightarrow$ &
   \textbf{Free Space} & 
   \textbf{Parking Car} &
   \textbf{Overtaking} &
   \textbf{Other Parking} \\
   True class $\downarrow$ &
	 & 
	 &
	 \textbf{Situation} &
	 \textbf{Vehicle} \\
\hline
  \textbf{Free Space} & 3288 & 1298 & 147 & 1947 \\
\hline
  \textbf{Parking Car} & 64 & 1926 & 56 & 137 \\
\hline
  \textbf{Overtaking Situation} & 0 & 22 & 29 & 15 \\
\hline
  \textbf{Other Parking Vehicle} & 2 & 10 & 0 & 48 \\
\hline

\end{tabular}
\egroup
}

\begin{center}
(b) filtered dataset\\
\end{center}

\caption{Confusion matrices of a naive bayes classifier applied on (a) the full dataset and (b) the filtered dataset.}
\label{table:cf_nb}
\end{table}

% \chapter*{Abbreviations}
\markright{Abbreviations}

\begin{description}
\setlength{\itemsep}{-11pt}
\setlength{\leftmargin}{900pt}

% C
\item[CAB] Compute Aggregate Broadcast (A computing model in parallel computing where computation is strictly partitioned in the three phases compute, aggregate and broadcast)

% I
\item[IaaS] Infrastructure as a Service (Cloud computing service layer)

% J
\item[JSF] Java Server Faces (Web technology in the arena of Java enterprise)
\item[JSP] Java Server Pages (Web technology available in Java Servlet containers)

% M
\item[MPI] Message Passing Interface (Standard for implementing parallel algorithms on shared-nothing infrastructures)

% O
\item[OSN] Online Social Network (An usually web-based online platform where friends, and acquaintances can connect and share information)

% P
\item[PaaS] Platform as a Service (Cloud computing service layer)

% S
\item[SaaS] Software as a Service (Cloud computing service layer)

% U
\item[UML] Unified Modeling Language
\item[URL] Uniform Resource Locator

% V
\item[VM] Virtual Machine

% W
\item[WAR] Web Application Archive

\end{description}


\input{structure/bibliography}

% curriculum vitae
%\input{structure/cv}
\end{document}
